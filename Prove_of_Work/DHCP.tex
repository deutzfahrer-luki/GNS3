\section{DHCP Konfiguration}
\subsection{Set IP Adresse}
\textit{\textbf{Aufgabenstellung:} Stelle ein, dass Fix $x.x.x.10$ - $x.x.x.99$ IP-Adressen vergibt}

\begin{itemize}
\item Öffne die DHCP-Config file auf dem Router, schreibe folgende Zeilen in die File und restart den service
\begin{verbatim}
vi /etc/config/dhcp
\end{verbatim}

\item \textbf{Gateway setzen}
\begin{verbatim}
config dhcp 'lan'
        option interface 'lan'
        option start '10'		## Change
        option limit '99'		## Change
        option leasetime '1h'  	## Change
        option dhcpv4 'server'
        option dhcpv6 'server'
        option ra 'server'
        option ra_slaac '1'
        list ra_flags 'managed-config'
        list ra_flags 'other-config'
\end{verbatim}

\item \textbf{Netzwerk Neustarten}
\begin{Huge} Frage TK \end{Huge}
\begin{verbatim}
/etc/init.d/network restart
/etc/init.d/network reload
\end{verbatim}
\end{itemize}

\newpage

\subsection{Fix IP Adresse auf MAC Adresse}
\textit{\textbf{Aufgabenstellung:} Stelle ein, dass eine fixe MAC-Adresse immer die gleiche IP zugewiesen bekommt}

\begin{itemize}
\item Öffne die DHCP-Config file auf dem Router, schreibe folgende Zeilen in die File und restart den service
\begin{verbatim}
vi /etc/config/dhcp
\end{verbatim}

\item \textbf{Gateway setzen}
\begin{verbatim}
## Add all
config host
        option name			'PC2'
        option mac			'00:50:79:66:68:01'
        option ip			'192.168.3.76'
        option leasetime 	'infinite'
\end{verbatim}

\item \textbf{Netzwerk Neustarten}
\begin{Huge} Frage TK \end{Huge}
\begin{verbatim}
/etc/init.d/network restart
/etc/init.d/network reload
\end{verbatim}
\end{itemize}