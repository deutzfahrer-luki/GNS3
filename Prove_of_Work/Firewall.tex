\section{Configuration der Firewall}
\subsection{Blockierung eines PCs}
\textbf{Aufgabenstellung}\\
\textit{Blockiere Zugang zum Netzwerk WAN für PC2 per ip}\\
\textbf{Voraussetzung}\\
DHCP-Server auf eine fixe IP über die MAC Adresse des PC2 fix zuweisen
\begin{itemize}
\item Öffne die DHCP-Config file, schreibe folgende Zeilen in die File und restart den service
\begin{verbatim}
config host
        option name				'PC2'
        option mac				'00:50:79:66:68:01'
        option ip				'192.168.3.133'
        option leasetime 		'infinite'
\end{verbatim}
\end{itemize}
\textbf{Firewall konfigurieren}
\begin{itemize}
\item öffne die Config files mit:
\begin{verbatim}
$ vi /etc/config/firewall
\end{verbatim}
\item schreibe folgende Zeilen rein:
\begin{verbatim}
## add all
config rule                                         
        option name             Block PC2           
        option proto            all                 
        option src              lan                 
        option dest             wan                 
        option target           REJECT              
        option src_ip           192.168.3.133  
\end{verbatim}
\item restart firewall
\begin{verbatim}
$ /etc/init.d/firewall restart
$ /etc/init.d/firewall reload
\end{verbatim}
\end{itemize}


\subsubsection{Wie kann ich die Netzsperre umgehen?}
\begin{itemize}
\item Wenn ich die IP fix blockiere kann ich wenn ich meine IP ändern kann (DHCP wurde keine fixe IP über MAC), kann ich die IP ändern und somit wird der PC nicht mehr erkannt.
\item Wenn ich die MAC-Adresse fix blockiere kann ich die Netzsperre nur mit Änderung der MAC-Adresse umgehen. (Kann man nicht auf GNS3 machen) 
\end{itemize}




\newpage

\subsection{Weiterleitung}
\textit{\textbf{Aufgabenstellung: } Stelle die Firewall so ein, dass die IP $192.168.3.76$ fix auf dem Port 80 weitergeleitet wird.}
\begin{itemize}
\item Öffne die DHCP-Config file, schreibe folgende Zeilen in die File und restart den service \\ Dass die mac des Debian Server die fixe IP bekommt.
\begin{verbatim}
## add all
config host
        option name 				'PC2'
        option mac 				'00:50:79:66:68:01'
        option ip 				'192.168.3.76'
        option leasetime			'infinite'
\end{verbatim}
\end{itemize}
\textbf{Firewall konfigurieren}
\begin{itemize}
\item öffne die Config files mit:
\begin{verbatim}
$ vi /etc/config/firewall
\end{verbatim}
\item schreibe folgende Zeilen rein:
\begin{verbatim}
## add all
config redirect                                        
        option dest             'lan'              
        option target           'DNAT'             
        option name             'Transmission'         
        option src              'wan'               
        option dest_ip          '192.168.3.76'         
        option dest_port        '80'                   
        option src_dport        '80'               
        option proto            'tcp'  
\end{verbatim}
\item restart firewall
\begin{verbatim}
$ /etc/init.d/firewall restart
$ /etc/init.d/firewall reload
\end{verbatim}
\end{itemize}
