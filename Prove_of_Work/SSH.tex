\section{SSH}
\begin{itemize}
	\item \textbf{Firewall entsprechend konfigurieren}
	\begin{itemize}
	\item öffne Firewall
	\item füge folgenden Eintrag hinzu
	\begin{verbatim}
	## Add all
	config rule                                            
        option name 'Allow-SSH'                        
        option src 'wan'                           
        option dest_port '22'                      
        option proto 'tcp'                             
        option target 'ACCEPT'
	\end{verbatim}	
	\end{itemize}		
	
	\item \textbf{Über Passwort Zugriff und ssh Zugriff aktivieren}
	\begin{itemize}
	\item Öffne die config-file
	\begin{verbatim}
	vi /etc/config/dropbear
	\end{verbatim}
	
	\item änder die ganze File auf: 
	\begin{verbatim}
	## Change all
	config dropbear
        option PasswordAuth 'on'
        option PubkeyAuth 'on'
        option RootPasswordAuth 'on'
        option Port         '22'
	\end{verbatim}
	\end{itemize}
	
	\item \textbf{ssh-key hinzufügen}
	\begin{itemize}
	\item Voraussetzung eines ssh key auf dem Host Rechner
	\begin{itemize}
	\item Hinzufügen (wenn nicht vorhanden)
	\begin{verbatim}
	ssh-keygen -t rsa -b 4096 -f ~/.ssh/id_rsa
	\end{verbatim}
	\end{itemize}	 
	\item ssh key auf Router kopieren
	\begin{verbatim}
	ssh-copy-id -i ~/.ssh/id_rsa.pub root@192.168.122.254
	\end{verbatim}
	\end{itemize}
		\newpage 
	\item \textbf{Über Passwort Zugriff deaktivieren, ssh Zugriff aktivieren lassen}
	\begin{itemize}
	\item Öffne die config-file
	\begin{verbatim}
	vi /etc/config/dropbear
	\end{verbatim}
	
	\item änder die ganze File auf: 
	\begin{verbatim}
	## Change all
	config dropbear
        option PasswordAuth 'off' ## Change
        option PubkeyAuth 'on'
        option RootPasswordAuth 'off' ## Change
        option Port         '22'
	\end{verbatim}
	\end{itemize}
	\item \textbf{Neu anmelden}
\end{itemize}
